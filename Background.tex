\section{Background}

% What are your technical skills, 
% education, experience, etc. 
% Especially make sure to explain with what level of 
% mathematics you are comfortable with and on what level you 
% would like to program.

\subsection{Education, Technical Background and experience}
I am currently a senior, set to graduate in a few months. Over the past few
years, I have studied various courses and gained experience in many areas. I
have mainly focused on competitive programming and have achieved several
successes:
\begin{itemize}
      \item Champion of
            \href{https://icpc.global/regionals/finder/ECPC-2025/standings}{ECPC 2024}
            (ICPC Egyptian Collegiate Programming Contest)
      \item Silver medalist with Rank 3 at
            \href{https://icpc.global/regionals/finder/ACPC-2025/standings}{ACPC 2024}
            (ICPC Africa \& Arab Collegiate Programming Championship)
      \item Qualified for the ICPC World Finals 2025 in Baku, Azerbaijan
      \item Finalist at the ICPC World Finals 2024 in Astana, Kazakhstan
\end{itemize}

You can view all my competitive programming achievements on my
\href{https://icpc.global/ICPCID/FRKWA0PTZMC8}{ICPCID}, or
\href{https://codeforces.com/profile/zezoo050}{Codeforces: zezoo050}.\\

I also joined \href{https://www.facebook.com/acmASCIS}{acmascis}, a competitive
programming community that trains and prepares students for contests. In my
junior year, I led the problem-setting and testing teams. We created and tested
problems not only for regional contests in Africa and the Arab region but also
for the community to help train students. I managed about 15 members, ensuring
that we met the community’s needs for problems. Additionally, I trained team
members so they could continue leading the community in the future.\\

Regarding mathematics, I am comfortable with the level needed to understand
algorithms and the math used in computer science. As part of my training for
competitive programming, I have studied various fields such as linear algebra,
discrete math, graph theory, and geometry.\\

For programming, I have experience at different levels. I have worked on many
projects, from building websites to developing low-level projects like database
engines and operating systems.\\

% Who are you? What makes you the best person to work on this particular project? Your personal motivation?

\subsection{About me \& motivation}
I have been interested in computer science since high school. When I started
college, people suggested I try competitive programming. As a naturally
competitive person, I found it enjoyable and learned a lot from it. After a
couple of years, I noticed improvements in my speed and problem-solving skills,
even for problems outside competitive programming. This training also made a
big difference in my ability to handle projects and my understanding to
theories. \\

My motivation for this project comes from two main sources. First, my passion
for graph theory drives me to explore its many applications. Second, I have a
strong desire to give back to the open source community. I have been using
Linux and other open source software for several years, and these resources
have helped me learn and grow. Now, I feel it is my turn to contribute. Google
summer of code is a very good opportunity to get into open source. I am still a
beginner and have a lot to learn, but I will do my best to contribute and
continue growing in this field.\\

% What platform and operating-system are you using on your computer? (SageMath development is done best on Linux, !MacOS, or Windows through WSL)
\subsection{Operating system \& Development environment}
I have been using Linux for almost four years and have gained a lot of
experience with it. Currently, my main operating system is Ubuntu.\\

% Are you or have you been engaged in other open-source projects?
\subsection{Open source contributions}
SageMath is my first open-source contribution. I have been using open-source
software as much as possible and have explored the code of multiple open-source
projects. However, my first direct engagement has been with SageMath. I was
able to build SageMath from source code and familiarized myself with the source
code. I was able to find a fix for an issue and opened a pull request
(\href{https://github.com/sagemath/sage/pull/39754}{\#39754}) and it got
approved. Additionally, reported a bug
(\href{https://github.com/sagemath/sage/issues/39756}{\#39756}).\\

% Do you code on your own pet projects?
\subsection{Pet projects}
I do code on my own pet projects from time to time when I get interested in
something, but currently I am more focused on GSoC and being able to contribute
to open source projects. Here are some of the projects that could be of an
interest:
\begin{itemize}
      \item \href{https://github.com/zezoo050/ImageEncryptCompress}{ImageEncryptCompress}: Compression and Decompression of images using Huffman tree, and also encryption and decryption of images.
      \item \href{https://github.com/zezoo050/spacechickensfmlgame}{spacechickensfmlgame}: space chicken game built using C++ and SFML.
      \item \href{https://github.com/zezoo050/GSoC25-SageMath-proposal}{GSoC25-SageMath-proposal}: I actually consider this proposal as one of my pet projects that I am proud of. 
      While I have been using $Latex$ for technical writing for competitive programming problems, I have never used it on such a scale as this proposal before.
\end{itemize}

% TODO: add this proposal as pet project

% Are you a SageMath user, how long have you known about/used SageMath?
\subsection{SageMath knowledge}
My knowledge of SageMath does not exceed a couple of months. I have been using
it for graph theory purposes, applying algorithms and visualizing graphs.

