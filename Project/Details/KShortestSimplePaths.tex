\subsubsection{K shortest simple paths}

\subparagraph{Current implementation}
The currently implemented methods are Yen's algorithm\cite{yen1971} and Feng's
algorithm (node classification)\cite{feng2014}. The method
$shortest\_simple\_paths$ will by default uses Yen's for undirected graphs and
Feng's for directed graphs.

\subparagraph{Proposed approaches}
To implement efficient algorithms for the k shortest simple paths problem, we
propose following the methods introduced in the paper “Finding the k Shortest
Simple Paths: Time and Space Trade-offs”\cite{kSSP2023}. This is a recent
paper, offering efficient optimizations for this problem. Furthermore, the fact
that David Coudert, the mentor of this GSoC project, is a co-author of the
paper makes this approach particularly well-suited for the project.

\subparagraph{Algorithms to be implemented}
\begin{itemize}
    \item \textbf{PNC} (Postponed Node Classification)\cite{kSSP2023}
          This algorithm performs well on graphs resembling road networks,
          where the structure tends to be sparse. It offer a faster performance in practice over the classic NC
          as It delays some of the calls until necessary, in order to avoid some of them.

          \begin{algorithm}[H]
              \caption{Postponed Node Classification (PNC)\cite{kSSP2023}}
              \begin{algorithmic}[1]
                  \Require A digraph \( D = (V, A) \), source \( s \in V \), sink \( t \in V \), integer \( k \)
                  \Ensure \( k \) shortest simple \( s-t \) paths

                  \State Let \( Candidate \gets \emptyset \) and \( Output \gets \emptyset \)
                  \State \( T \gets \) an SP in-branching of \( D \) rooted at \( t \)
                  \State Add \( (P_{st}(T), \omega(P_{st}(T)), 0, 1) \) to \( Candidate \)

                  \While{\( Candidate \neq \emptyset \) and \( |Output| < k \)}
                  \State \((P = (s, u_1, \cdots, t), \omega(P), i, \zeta) \gets \text{extract shortest element from } Candidate\)
                  \State \(\pi \gets (s, u_1, \cdots, u_{i-1})\)
                  \State \( Dev_{old} \gets \{e = (u_i, v) \mid \exists \text{ path in } Output \text{ with prefix } \pi.e \} \)

                  \If{\(\zeta = 1\) \textbf{(\( P \) is simple)}}
                  \State Add \( P \) to \( Output \)
                  \State \(\lambda \gets \text{vertex labeling of } D \text{ with respect to } P \text{ and } T\)

                  \For{each vertex \( u_j \) in \( (u_i, \cdots, t) \)}
                  \State \((u_j, v_{LB}) \gets \text{arc in } A \setminus Dev_{old} \text{ with min } \delta \text{ at } u_j\)
                  \State \( P_{LB} \gets (s, \cdots, u_j, v_{LB}, P^T_{v_{LB}t}) \)
                  \State \(\zeta' \gets 0\)
                  \If{\(\lambda(v_{LB}) > j\) \textbf{(\( P_{LB} \) is simple)}}
                  \State \(\zeta' \gets 1\)
                  \EndIf
                  \State Add \( (P_{LB}, \omega(P_{LB}), j, \zeta') \) to \( Candidate \)
                  \EndFor

                  \Else
                  \State Compute shortest \( u_i-t \) path \( Q \) in \( D' = (V \setminus \pi, A \setminus Dev_{old}) \)
                  \If{\( Q \) exists}
                  \State Add \( (P' = \pi.Q, \omega(P'), i, 1) \) to \( Candidate \)
                  \EndIf
                  \EndIf
                  \EndWhile

                  \State \Return \( Output \)
              \end{algorithmic}
          \end{algorithm}

    \item \textbf{PSB} (Parsimonious Sidetrack-Based)\cite{kSSP2023}
          This variant is significantly more efficient for complex networks
          (e.g., social networks, web graphs) where the degrees of the node is higher
          and diameter tends to be shorter. It offers a trade-off compared to the
          PNC as it uses more memory but is faster on this type of graphs.

          \begin{algorithm}[H]
              \caption{Sidetrack Based (SB) Algorithm for the \(k\)SSP (which the PSB is based upon)\cite{SB2016}}
              \begin{algorithmic}[1]
                  \Require A digraph \(D = (V, A)\), source \(s \in V\), sink \(t \in V\), integer \(k\)
                  \Ensure \(k\) shortest simple \(s\)-\(t\) paths

                  \State Let \(Candidate \gets \emptyset\) and \(Output \gets \emptyset\)
                  \State \(T_0 \gets\) an SP in-branching of \(D\) rooted at \(t\) containing \(s\)
                  \State Add \(\big((T_0), \omega(P_{st}(T_0)), \zeta = 1\big)\) to \(Candidate\)

                  \While{\(Candidate \neq \emptyset\) and \(|Output| < k\)}
                  \State \(\mu = \big(\varepsilon = ((T_0, e_0, \cdots, T_h, e_h = (u_h, v_h), T_{h+1}), lb, \zeta\big)) \gets\) a shortest element in \(Candidate\)

                  \If{\(\zeta = 1\)}
                  \State Extract \(\mu\) from \(Candidate\) and add \(\varepsilon\) to \(Output\)
                  \For{every deviation \(e = v_jv'\) with \(v_j \in P^{T_{h+1}}_{v_i,t}\)}
                  \State \(\varepsilon' \gets (T_0, e_0, \cdots, T_h, e_h, T_{h+1}, e, T_{h+1})\)
                  \State \(lb' \gets lb - \omega(P^{T_{h+1}}_{v_j,t}) + \omega(e) + \omega(P^{T_{h+1}}_{v't})\)
                  \If{\(\varepsilon'\) represents a simple path}
                  \State Add \(\mu' = (\varepsilon', lb', \zeta = 1)\) to \(Candidate\)
                  \Else
                  \State \(T' \gets\) the name of an SP in-branching of \(D_j(P)\) \Comment{\(T'\) is not computed yet}
                  \State Add \(\mu'' = \big(\varepsilon'' = (T_0, e_0, \cdots, T_h, e_h, T_{h+1}, e, T'), lb', \zeta = 0\big)\) to \(Candidate\)
                  \EndIf
                  \EndFor
                  \Else
                  \If{\(T_{h+1}\) has not been computed yet}
                  \State Compute \(T_{h+1}\), an SP in-branching of \(D_h(P)\)
                  \EndIf
                  \State \(\mu' = \big(\varepsilon' = (T_0, e_0, \cdots, T_h, e_h, T_{h+1}), lb + \omega(P^{T_{h+1}}_{v_i,t}) - \omega(P^{T_h}_{v_i,t}), \zeta = 1\big)\)
                  \State Add \(\mu'\) to \(Candidate\)
                  \EndIf
                  \EndWhile

                  \State \Return \(Output\)
              \end{algorithmic}
          \end{algorithm}
\end{itemize}

\subparagraph{Selection algorithm} As discussed in the paper, the optimal algorithm depends heavily on the
properties of the input graph. What I propose:

Analyzing the graph (e.g., average degree, diameter) and Automatically
selecting between PNC and PSB based on the graph’s properties:
\begin{itemize}
    \item Use PNC for sparse, road-like graphs
    \item Use PSB for dense, complex networks
\end{itemize}

\begin{figure}[ht]
    \centering

    \begin{tikzpicture}
        [
        node distance=1cm and 1cm,
        every node/.style={draw, rectangle, rounded corners, align=center},
        arrow/.style={-{Stealth}, thick}
        ]

        \node (kssp) {kSSP};
        \node (complex) [below left=of kssp] {Complex network};
        \node (road) [below right=of kssp] {Road network};

        \node (psb) [below=of complex] {PSB};

        \node (pnc) [below=of road] {PNC};

        % Edges
        \draw [arrow] (kssp) -- (complex);
        \draw [arrow] (kssp) -- (road);

        \draw [arrow] (complex) -- (psb);

        \draw [arrow] (road) -- (pnc);

    \end{tikzpicture}
    \caption{Selection decisions for kSSP}
\end{figure}

Both algorithms natively support weighted directed graphs, and can be easily
extended to undirected graphs and unweighted graphs.

\subparagraph{Deliverables}
These are the tasks that are expected to be delivered for this part of the
project.

\begin{itemize}
    \item PNC (Postponed Node Classification)
    \item PSB (Parsimonious Sidetrack-Based)
    \item Selection algorithm
\end{itemize}

Each and every task will contain the following

\begin{itemize}
    \item Core implementation
    \item Testing
    \item Documentation
\end{itemize}

All methods will follow a consistent input/output format:
\begin{itemize}
    \item \textbf{Input}: Parameters specifying the paths properties (e.g., starting vertices, ending vertices, weight thresholds).
    \item \textbf{Output}: Iterators yielding paths in ascending order of weight/length.
\end{itemize}

\textbf{Note}: Neither the current implementation nor the proposed one will consider a
parameter k (the number of shortest paths to retrieve). Instead, the iterator
will continue computing paths until no more can be found.

